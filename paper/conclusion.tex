\section{Conclusions}

% We were able to communicate any team that could be made using 10 different languages, having 10 possible configurations and 10 possible tasks to performs. All this without knowing any of these parameters beforehand but only the set of possibles. To our knowledge it is the first time these three unknown are considered simultaneoulsy in a ad hoc settings.

The results presented in this paper show that an ad hoc agent can integrate into a team without knowing in advance the task, its role, and the communication protocol of the team. To our knowledge it is the first time that these three aspects are considered simultaneously in an ad hoc setting. Notably, we believe that this is the first paper to address ambiguous communication protocols in ad hoc teams. We used exact inference to infer in only a few iterations the correct team configuration. As a result, the performance of the team was barely impacted.

But considering that many hypotheses is costly, and the approach presented in this paper computationally expensive (see Figure~\ref{fig:comptime}). An important challenge for the future is to find ways to approximate this process while minimizing the impact on the performance of the team. A potential avenue is to consider a sampling strategy, evaluating only a subset of all possible domains each step.

Finally, our results show that the default team we built is not optimal. Indeed an ad hoc agent, which is not always taking the action the agent it replaces would have chosen, can on average achieve similar performances. If the pre-coordinated team was optimal, we would expect the performance of the ad hoc team to be ``delayed'' -- having the same slope but loosing some important steps in the beginning. Therefore, it is likely that a more advanced planning method for the ad hoc agent (see \cite{barrett2011empirical}) could improve the performance of the default team.

% In this work we will proceed with the exact inference but approximating this update rule is part of the future work.
